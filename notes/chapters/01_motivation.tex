\chapter{Motivation}

\section{Modern motivation}

People study symmetric functions because they want to:
\begin{itemize}
    \item understand symmetric groups;
    \item understand general linear groups;
    \item do combinatorics and don't care about the rest;
    \item do algebraic geometry.
\end{itemize}

\section{Historical motivation}

Before representation theory, people were interested in question like the following:

\begin{question}
    Let the special linear group \(\SL{n}\) acting on the matrix space \(\mathcal{M}_{n \times m}\).
    Also let the special linear group \(\SL{n}\) act on the polynomial space in variables \(x_{ij}\) by
    \begin{equation}
        A \cdot p(v) = p(A^{-1}v).
    \end{equation}
    Which polynomials are invariant under such linear change of coordinates?
\end{question}

Constant polynomials are invariant, but what else?
By the defining property of the special linear group that the determinant of its elements is 1,
the determinant polynomials of the \(\binom{M}{n}\) submatrices are invariant.
These polynomials are the generators of the algebra of invariants of the special linear group acting on the polynomial space.
This fact is known as \vocab{Hilbert's fundamental theorem of invariant theory}.

Some things are nice about this algebra.
For example, the invariantes are finitely generated by these \(\binom{m}{n}\) polynomials.
The relations between these generators are also known,
and are called \vocab{Plücker relations}.

People were excited about this that they wanted to generalize this to other groups.
In general, let \(G\) be a group acting on a finite-dimensional vector space \(V\) over \(\mathbb{C}\), and let \(\pi \colon G \mapsto \operatorname{GL}(V)\) be a group homomorphism.
Then, we get an action on \(\mathbb{C}[V]\), the space of polynomials functions on \(V\),%
\footnote{To be more concrete, choose a basis for \(V\) and write a polynomial as a polynomial in the coordinates of the basis.}
by 
\begin{equation}
    g \cdot p(v) = p(\pi(g)^{-1}v).
\end{equation}
We get invariants \(\mathbb{C}[V]^G\), the space of polynomials invariant under the action of \(G\).
People did all sorts of examples for \(V\)'s and \(G\)'s.

What kinds of questions can we ask?

\begin{question}
    Is \(\mathbb{C}[V]^G\) finitely generated?
\end{question}

For most of the examples, the answer is yes.

\begin{question}
    If so, what are the minimal generators and what are the relations between them?
\end{question}

People did it all over the place and found relations similar to the Plücker relations.

We will spend the term studying the best success story of this program,
and all of other examples will be trying to mimic this success story.

For us, the group will be the symmetric group \(G = S_n\) acting on the vector space \(V = \mathbb{C}^n\), with \(G\) acting as the permutation matrices.
Then, \(\mathbb{C}[V]^G = \Sym_n\) is our main object of study, the space of symmetric functions.%
\footnote{Fun fact: the symmetric group is named ``symmetric'' after the symmetric functions.}

We'll see that it has the best possible properties that we could hope for.
It is finitely generated by \(n\) elements, and they have no relations between them.
We can talk about basis of \(\Sym_n\).

Also, \(\Sym_n\) is basically a representation ring of the general linear group \(\GL{n}\).
It is also basically a representation ring of the symmetric group \(S_n\).
It is also basically the cohomology ring of the Grassmannian.

It also solves other problems, like
\begin{question}[Horn problem]
    If \(A\), \(B\), \(C\) are Hermitian matrices
    with \(A + B = C\),
    how do the eigenvalues of \(C\) depend on the eigenvalues of \(A\) and \(B\)?
\end{question}
