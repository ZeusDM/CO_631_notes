\chapter{Symmetric Functions}

Let \(\sym_\infty\) be the group of bijections from \(\nonnegatives\) to itself that fix all but finitely many elements, under composition.
Let \(\rationals[[x_1, x_2, \ldots]]_{\mathsf{gr}}\)
be the subalgebra consisting of the power series in \(x_1, x_2, \ldots\)
of bounded degree.

For example,
\begin{align*}
    x_1 + x_2 + x_3 + \cdots                   & \in \rationals[[x_1, x_2, \ldots]]_{\mathsf{gr}}, \\
    x_1x_2 + x_1x_3 + \cdots + x_2x_3 + \cdots & \in \rationals[[x_1, x_2, \ldots]]_{\mathsf{gr}},
\end{align*}
but
\begin{align*}
    x_1 + x_1^2 + x_1^3 + \cdots \notin \rationals[[x_1, x_2, \ldots]]_{\mathsf{gr}},
    x_1x_2x_3x_4\cdots \notin \rationals[[x_1, x_2, \ldots]]_{\mathsf{gr}}.
\end{align*}

Note that \(\rationals[[x_1, x_2, \ldots]]_{\mathsf{gr}}\) is a graded \(\rationals\)-algebra,
however its homogeneous components are infinite-dimensional,
hence it has no Hilbert series.

\begin{definition}
    The \vocab{algebra of symmetric functions} is
    \[
        \Sym = \rationals[[x_1, x_2, \ldots]]_{\mathsf{gr}}^{\sym_\infty},
    \]
\end{definition}

Note that there are projections \(\Sym(n) \to \Sym(n-1)\) for all \(n\),
where \(x_n\) is sent to \(0\) and the other variables are fixed.
Thus, we have a sequence of maps
\[
    \cdots \to \Sym(n) \to \Sym(n-1) \to \cdots \to \Sym(2) \to \Sym(1) \to \rationals,
\]
and the algebra of symmetric functions \(\Sym\) is the inverse limit of this sequence; whatevers that means.

There is a projection \(\Sym \to \Sym(n)\) that sends \(x_i\) to \(0\) for all \(i > n\), and the other variables are fixed.

Given a partition \(\lambda \in \Par\),
the \vocab{monomial symmetric function} \(m_\lambda \in \Sym\) is defined by
\[
    m_\lambda = \sum_{\sort{\alpha} = \lambda} x^\alpha.
\]
The set of monomial symmetric functions
\[
    \{ m_\lambda \mid \lambda \in \Par \}
\]
is a basis for \(\Sym\).

Any formula that holds in \(\Sym\) will hold in \(\Sym(n)\) after sending \(m_\lambda\) to \(m_\lambda(x_1, x_2, \ldots, x_n)\) if \(\ell(\lambda) \leq n\), and to \(0\) otherwise.
In some sense, the converse of this statement is also true: any formula that holds in \(\Sym(n)\) for all \(n\) will hold in \(\Sym\).

We might also want to define the \vocab{formal completion} of \(\Sym\) as
\begin{equation*}
    \widehat\Sym = \rationals[[x_1, x_2, \ldots]]^{\sym_\infty}.
\end{equation*}
For example,
\begin{equation*}
    m_1 + m_2 + m_3 + \cdots \in \widehat\Sym.
\end{equation*}

Back to the algebra of symmetric functions,
\(\Sym\) is a graded \(\rationals\)-algebra,
where the homogeneous component of degree \(d\) is \(\Sym_d\).
The dimension of \(\Sym_d\) is the number of partitions of \(d\),
and therefore the Hilbert series of \(\Sym\) is
\begin{equation*}
    \Hilb(\Sym; t) = \prod_{i=1}^\infty \frac{1}{1-t^i}.
\end{equation*}

Consider \(N = \rationals[x_1, x_2, \ldots]\), the ring of polynomials in infinitely many variables, which is also a graded \(\rationals\)-algebra.
But \(\dim N_1 = \infty\), so \(N\) has no Hilbert series.
Consider \(N' = N\) with nonstandard grading where \(\deg(x_i) = i\) for all \(i \in \nonnegatives\).
Then, \(\dim N'_d\) is the number of partitions of \(d\), so
\begin{equation*}
    \Hilb(N'; t) = \prod_{i=1}^\infty \frac{1}{1-t^i} = \Hilb(\Sym; t).
\end{equation*}
This makes us expect that \(\Sym\) to look like \(N'\).

How do we explain this? We find generators of \(\Sym\), one in each degree, with no relations between them.
There are three simple answers to this question:
\begin{itemize}
    \item \vocab{power sum symmetric functions} \(p_k = \sum_{i=1}^\infty x_i^k\),
    \item \vocab{elementary symmetric functions} \(e_k = \sum_{1 \leq i_1 < i_2 < \cdots < i_k} x_{i_1}x_{i_2}\cdots x_{i_k}\),
    \item \vocab{complete homogeneous symmetric functions} \(h_k = \sum_{1 \leq i_1 \leq i_2 \leq \cdots \leq i_k} x_{i_1}x_{i_2}\cdots x_{i_k}\).
\end{itemize}

\section{Power Sum Symmetric Functions}

The \vocab{\(n\)\textsuperscript{th} power sum symmetric function \(p_n \in \Sym\)} is defined by
\begin{equation}
    p_n = \sum_{i=1}^\infty x_i^n.
\end{equation}
Given a partition \(\lambda \in \Par\), define \(p_\lambda = p_{\lambda_1}p_{\lambda_2}\cdots p_{\lambda_{\ell(\lambda)}}\).

\begin{theorem} \label{thm:pn_sym}
    The four following statements are true:
    \begin{enumerate}
        \item \label{item:pl_symd_basis}
              The set \(\{p_\lambda : \lambda \in \Par(d)\}\) forms a basis for \(\Sym_d\).
        \item \label{item:pl_sym_basis}
              The set \(\{p_\lambda : \lambda \in \Par\}\) forms a basis for \(\Sym\).
        \item \label{item:pn_sym_generates}
              The set \(\{p_i : i \in \positives\}\) generates \(\Sym\) and is algebraically independent.
        \item \label{item:pn_sym_graded_isom}
              The map \(N' \to \Sym\) obtained by the extension of the map \(x_i \mapsto p_i\) is an isomorphism of graded \(\rationals\)-algebras.
    \end{enumerate}
\end{theorem}

\begin{example}[Theorem~\ref{thm:pn_sym}\ref{item:pl_symd_basis} for \(d = 3\)]
    There are three partitions of \(3\), namely \(\composition{3}\), \(\composition{2, 1}\), and \(\composition{1, 1, 1}\).
    The corresponding power sum symmetric functions are
    \begin{align*}
        p_{\composition{3}}       & = p_3 = x_1^3 + x_2^3 + x_3^3 + \cdots,                                \\
        p_{\composition{2, 1}}    & = p_2p_1 = (x_1^2 + x_2^2 + x_3^2 + \cdots)(x_1 + x_2 + x_3 + \cdots), \\
        p_{\composition{1, 1, 1}} & = p_1^3 = (x_1 + x_2 + x_3 + \cdots)^3,
    \end{align*}
    which can be expanded in terms of monomial symmetric functions as
    \begin{align*}
        p_{\composition{3}}       & = m_{\composition{3}}                                                         \\
        p_{\composition{2, 1}}    & = m_{\composition{3}} + m_{\composition{2, 1}}                                \\
        p_{\composition{1, 1, 1}} & = m_{\composition{3}} + 3m_{\composition{2, 1}} + 6m_{\composition{1, 1, 1}}.
    \end{align*}
    Hence,
    \begin{equation*}
        \begin{bsmallmatrix}
            p_{\composition{1, 1, 1}} \\ p_{\composition{2, 1}} \\ p_{\composition{3}}
        \end{bsmallmatrix}
        =
        \begin{bsmallmatrix}
            6 & 3 & 1 \\
            0 & 1 & 1 \\
            0 & 0 & 1
        \end{bsmallmatrix}
        \begin{bsmallmatrix}
            m_{\composition{1, 1, 1}} \\ m_{\composition{2, 1}} \\ m_{\composition{3}}
        \end{bsmallmatrix}.
    \end{equation*}
    Since
    \(
    \begin{bsmallmatrix}
        6 & 3 & 1 \\
        0 & 1 & 1 \\
        0 & 0 & 1
    \end{bsmallmatrix}
    \)
    is an invertible matrix and \(\{m_{\composition{3}}, m_{\composition{2, 1}}, m_{\composition{1, 1, 1}}\}\) is a basis for \(\Sym_3\),
    it follows that \(\{p_{\composition{3}}, p_{\composition{2, 1}}, p_{\composition{1, 1, 1}}\}\) is a basis for \(\Sym_3\).

    Note that \(\{p_{\composition{3}}, p_{\composition{2, 1}}, p_{\composition{1, 1, 1}}\}\) is \emph{not} a \(\mathbb{Z}\)-basis for the integer symmetric functions, as the matrix has determinant \(6\).
\end{example}

\begin{proof}[Proof of Theorem~\ref{thm:pn_sym}(a)]
    Let \(d\) be an integer and \(\lambda \in \Par(d)\).
    Note that
    \begin{align*}
        p_\lambda
         & = p_{\lambda_1}p_{\lambda_2}\cdots p_{\lambda_{\ell(\lambda)}}            \\
         & = \left( \prod_i j_i! \right) m_\lambda + \sum_{\substack{\mu \in \Par(d) \\ \lambda \lessref \mu}} R_{\lambda\mu}m_\mu,
    \end{align*}
    where \(R_{\lambda\mu}\) are positive integers.%
    \footnote{A future homework problem might ask to compute \(R_{\lambda\mu}\).}
    Take a total order that extends the refinement order on \(\Par(d)\), and write two vectors \((p_\lambda)\) and \((m_\lambda)\) in this order.
    Then, the matrix that maps \((\mu_\lambda)\) to \((p_\lambda)\) is upper triangular with positive diagonal entries, hence invertible.
    Finally, since \(\{m_\lambda : \lambda \in \Par(d)\}\) is a basis for \(\Sym_d\), it follows that \(\{p_\lambda : \lambda \in \Par(d)\}\) is a basis for \(\Sym_d\).
\end{proof}

Note that the \(p\)-basis is \(m\)-positive, i.e., the coefficients of the \(p\)-basis in the \(m\)-basis are nonnegative integers,
and \(m\)-integral, i.e., the coefficients of the \(p\)-basis in the \(m\)-basis are integers.

\section{Elementary Symmetric Functions}

The \vocab{\(n\)\textsuperscript{th} elementary symmetric function \(e_n \in \Sym\)} is defined by
\begin{equation}
    e_n = \sum_{1 \leq i_1 < i_2 < \cdots < i_n} x_{i_1}x_{i_2}\cdots x_{i_n}.
\end{equation}
Given a partition \(\lambda \in \Par\), define \(e_\lambda = e_{\lambda_1}e_{\lambda_2}\cdots e_{\lambda_{\ell(\lambda)}}\).

\begin{theorem}[Fundamental Theorem of Symmetric Functions] \label{thm:en_sym}
    The four following statements are true:
    \begin{enumerate}
        \item \label{item:el_symd_basis}
              The set \(\{e_\lambda : \lambda \in \Par(d)\}\) forms a basis for \(\Sym_d\).
        \item \label{item:el_sym_basis}
              The set \(\{e_\lambda : \lambda \in \Par\}\) forms a basis for \(\Sym\).
        \item \label{item:en_sym_generates}
              The set \(\{e_i : i \in \positives\}\) generates \(\Sym\) and is algebraically independent.
        \item \label{item:en_sym_graded_isom}
              The map \(N' \to \Sym\) obtained by the extension of the map \(x_i \mapsto e_i\) is an isomorphism of graded \(\rationals\)-algebras.
    \end{enumerate}
\end{theorem}

A \vocab{contingency table} is a function \(A \colon \positives \times \positives \to \nonnegatives\) such that all but finitely many entries are non-zero.
Let \(\operatorname{row}(A)\) be the sequence of row sums of \(A\),
and let \(\operatorname{col}(A)\) be the sequence of column sums of \(A\).
Note that the sum of \(\operatorname{row}(A)\) is equal to the sum of \(\operatorname{col}(A)\), hence they are weak compositions of the same integer.
See Figure~\ref{fig:contingency_table} for an example of a contingency table.

\begin{figure}[htbp]
	\begin{equation*}
		\arraycolsep=7pt
		\def\arraystretch{1.3}
		\begin{array}{c|ccccc}
			& \textcolor{red}{4} & \textcolor{red}{6} & \textcolor{red}{5} & \textcolor{red}{0} & \cdots \\ \hline
			\textcolor{red}{6} & 1      & 4      & 3      & 0      & \cdots \\
			\textcolor{red}{3} & 0      & 2      & 1      & 0      & \cdots \\
			\textcolor{red}{4} & 3      & 0      & 1      & 0      & \cdots \\
			\textcolor{red}{0} & 0      & 0      & 0      & 0      & \cdots \\
			\vdots             & \vdots & \vdots & \vdots & \vdots & \ddots
		\end{array}
	\end{equation*}
	\caption{A contingency table (in black) \(A\) with \(\operatorname{row}(A) = \composition{6, 3, 4}\) and \(\operatorname{col}(A) = \composition{4, 6, 5}\) (in red).}
	\label{fig:contingency_table}
\end{figure}

Given weak compositions \(\alpha, \beta\),
let \(M_{\alpha\beta}\) be the number of contingency tables with entries in \(\{0, 1\}\) such that \(\operatorname{row}(A) = \alpha\) and \(\operatorname{col}(A) = \beta\).

We start with some facts about \(M_{\alpha\beta}\), which are used later without reference.

\begin{fact} \label{fact:Mab-same-sum}
	If \(\alpha, \beta\) are weak compositions of different integers, then \(M_{\alpha\beta} = 0\).
\end{fact}
\begin{proof}
	If \(A\) is a contingency table with \(\operatorname{row}(A) = \alpha\) and \(\operatorname{col}(A) = \beta\),
	then the sum of \(\alpha\) is equal to the sum of \(\beta\) because both are equal to the sum of all entries of \(A\).
\end{proof}

\begin{fact} \label{fact:Mab-Mba}
	For any weak compositions \(\alpha, \beta\),
	\(M_{\alpha\beta} = M_{\beta\alpha}\).
\end{fact}
\begin{proof}
	Reflection by the main diagonal of a contingency table is a bijection between the objects counted by \(M_{\alpha\beta}\) and \(M_{\beta\alpha}\).
\end{proof}

\begin{fact} \label{fact:Mab-rearrangement}
	If \(\alpha\) is a rearrangement of \(\alpha'\) and \(\beta\) is a rearrangement of \(\beta'\),
	then \(M_{\alpha\beta} = M_{\alpha'\beta'}\).
\end{fact}
\begin{proof}
	Let \(w\) be the permutation that rearranges \(\alpha\) to \(\alpha'\).
	The map that sends a contingency table \(A\) to another contingency table \(A'\) by applying \(w\) to the rows of \(A\) is a bijection between the objects counted by \(M_{\alpha\beta}\) and \(M_{\alpha'\beta}\).
	Applying the same argument to the columns shows that \(M_{\alpha\beta} = M_{\alpha'\beta'}\), as desired.
\end{proof}

\begin{corollary} \label{cor:Mab-Mlm}
	If \(\alpha\) is a rearrangement of a partition \(\lambda\) and \(\beta\) is a rearrangement of a partition \(\mu\),
	then \(M_{\alpha\beta} = M_{\lambda\mu}\).
\end{corollary}

Define the weight of a contingency table \(A\) to be the monomial
\begin{equation}
	\operatorname{wt}(A)
	= \prod_{i, j \geq 1} (x_iy_j)^{A(i, j)}
	= x^{\operatorname{row}(A)}y^{\operatorname{col}(A)},
\end{equation}

\begin{lemma} \label{lem:generating-function-01-contingency-tables-compositions}
	Let \(x = (x_1, x_2, \ldots)\) and \(y = (y_1, y_2, \ldots)\) be sequences of variables.
	Then,
	\begin{equation}
		\prod_{i, j \geq 1} (1 + x_iy_j)
		= \sum_{\substack{\text{contingency tables } A \\ \text{with entries in } \{0, 1\}}} \operatorname{wt}(A)
		= \sum_{\substack{\text{weak compositions} \\ \alpha, \beta}} M_{\alpha\beta} x^\alpha y^\beta.
	\end{equation}
\end{lemma}

\begin{proof}
Note that each monomial in the expansion of the power series
\begin{equation}
	\prod_{i, j \geq 1} (1 + x_iy_j)
	= \prod_{i, j \geq 1} \left( (x_iy_j)^0 + (x_iy_j)^1 \right)
\end{equation}
corresponds to a choice of \(0\) or \(1\) for the exponent of each \(x_iy_j\), which corresponds to a contingency table with entries in \(\{0, 1\}\).
Therefore,
\begin{equation}
	\prod_{i, j \geq 1} (1 + x_iy_j)
	= \sum_{\substack{\text{contingency tables } A \\ \text{with entries in } \{0, 1\}}} \operatorname{wt}(A) 
	= \sum_{\substack{\text{weak compositions} \\ \alpha, \beta}} M_{\alpha\beta} x^\alpha y^\beta, 
\end{equation}
where the last equation is obtained by computing the coefficient of each monomial.
\end{proof}

\begin{proposition} \label{prop:generating-function-01-contingency-tables-partitions}
	Let \(x = (x_1, x_2, \ldots)\) and \(y = (y_1, y_2, \ldots)\) be sequences of variables.
	Then,
	\begin{equation}
		\prod_{i, j \geq 1} (1 + x_iy_j) = \sum_{\substack{\text{partitions} \\ \lambda, \mu}} M_{\lambda\mu} m_\lambda(x) m_\mu(y).
	\end{equation}
\end{proposition}

\begin{proof}
	We have
\begin{equation}
	\prod_{i, j \geq 1} (1 + x_iy_j)
	= \sum_{\substack{\text{weak compositions} \\ \alpha, \beta}} M_{\alpha\beta} x^\alpha y^\beta
	= \sum_{\substack{\text{partitions} \\ \lambda, \mu}} M_{\lambda\mu} m_\lambda(x) m_\mu(y),
\end{equation}
where the first equality is by Lemma~\ref{lem:generating-function-01-contingency-tables-compositions},
and the second equality is obtained by collecting all monomials whose exponents are rearrangements of the same partition.
\end{proof}

\begin{proposition} \label{prop:el-expansion-in-m-basis}
	Let \(\lambda\) be a partition.
	Then,
	\begin{equation}
		e_\lambda = \sum_{\text{partitions } \mu} M_{\lambda\mu} m_\mu.
	\end{equation}
\end{proposition}

\begin{proof}
	Note that 
	\begin{equation}
		e_k(x)
		= \sum_{\substack{\{0, 1\}\text{-sequences } \alpha \\ \text{with sum } k}}
			x^\alpha
	\end{equation}
	Hence,
	\begin{align}
		e_\lambda(x)
		= \prod_{i\geq 1} e_{\lambda_i}
		&= \prod_{i\geq 1}
			\sum_{\substack{\text{weak compositions } \alpha \\ \text{with entries in } \{0, 1\} \\ \text{and sum } \lambda_i}}
				x^{\alpha^{(i)}} \\
		&= \sum_{\substack{\text{contingency tables } A \\ \text{with entries in } \{0, 1\} \\ \text{and column sum } \lambda}}
			x^{\operatorname{row}(A)}.
	\end{align}
	By Lemma~\ref{lem:generating-function-01-contingency-tables-compositions},
	\(e_\lambda(x)\) is the coefficient of \(y^\lambda\) in \(\prod_{i, j \geq 1} (1 + x_iy_j)\) as a power series in \(y\) with coefficients in \(\rationals[[x_1, x_2, \ldots]]\),
	that is,
	\begin{equation}
		e_\lambda(x) = \sum_{\text{weak compositions } \alpha} M_{\alpha\lambda} x^\alpha = \sum_{\text{partitions } \mu} M_{\mu\lambda} m_\mu(x),
	\end{equation}
	which completes the proof.
\end{proof}

The following corollary is a direct consequence of Propositions~\ref{prop:generating-function-01-contingency-tables-partitions} and~\ref{prop:el-expansion-in-m-basis}.

\begin{corollary}
	Let \(x = (x_1, x_2, \ldots)\) and \(y = (y_1, y_2, \ldots)\) be sequences of variables.
	Let \(\lambda\) be a partition.
	Then,
	\begin{equation}
		\prod_{i, j \geq 1} (1 + x_iy_j) = \sum_{\lambda \in \Par} e_\lambda(x) m_\lambda(y).
	\end{equation}	
\end{corollary}

\begin{lemma}
    Let \(\lambda, \mu \in \Par(d)\).
    Then, \(M_{\lambda\mu} = 0\) if \(\mu \not\leq \tilde{\lambda}\),
    where \(\tilde{\lambda}\) is the conjugate of \(\lambda\).
    Moreover, \(M_{\lambda\tilde{\lambda}} = 1\).
\end{lemma}

\begin{proof}
	Suppose \(M_{\lambda\mu} \neq 0\).
	Then, there is a contingency table \(A\) with entries in \(\{0, 1\}\) such that \(\operatorname{row}(A) = \lambda\) and \(\operatorname{col}(A) = \mu\).
	Let \(B\) be the contingency table obtained by left-justifying the ones in \(A\).
	Then, \(\operatorname{row}(B) = \lambda\) and \(\operatorname{col}(B) = \tilde{\lambda}\).

	Fix \(i\).
	Note that
	\begin{align}
		\tilde{\lambda}_1 + \dots + \tilde{\lambda}_i
		 & = \text{number of ones in the first \(i\) columns of \(B\)}    \\
		 & \geq \text{number of ones in the first \(i\) columns of \(A\)}
		\geq \mu_1 + \dots + \mu_i.
	\end{align}
	Therefore, \(\tilde{\lambda} \geq \mu\).

	It is straightforward to see that \(M_{\lambda\tilde{\lambda}} = 1\), with the only possible contingency table having ones in the entries of the Young diagram of \(\lambda\) (in the English notation).
\end{proof}

\begin{proof}[Proof of Theorem~\ref{thm:en_sym}\ref{item:el_symd_basis}]
    Let \(d\) be an integer and \(\lambda \in \Par(d)\).
    Note that
    \begin{equation*}
        e_{\tilde{\lambda}}
        = m_\lambda + \sum_{\substack{\mu \in \Par(d) \\ \mu < \lambda}} M_{\lambda\mu} m_\mu.
    \end{equation*}
    Take a total order that extends the dominance order on \(\Par(d)\),
    and write two vectors \((e_{\tilde{\lambda}})\) and \((m_\lambda)\) in this order.
    Then, the matrix that maps \((m_\mu)\) to \((e_{\tilde{\lambda}})\) is upper triangular with positive diagonal entries, hence invertible.
    Finally, since \(\{m_\lambda : \lambda \in \Par(d)\}\) is a basis for \(\Sym_d\), it follows that \(\{e_{\tilde{\lambda}} : \lambda \in \Par(d)\}\) is a basis for \(\Sym_d\).

    Moreover, since the diagonal entries of the matrix are all \(1\), this matrix has an inverse with integer entries, and consequently \(\{e_{\tilde{\lambda}} : \lambda \in \Par(d)\}\) is a \(\mathbb{Z}\)-basis for the integer symmetric functions.
\end{proof}


\section{Complete Homogeneous Symmetric Functions}

The \vocab{\(n\)\textsuperscript{th} complete homogeneous symmetric function \(h_n \in \Sym\)} is defined by
\begin{equation}
    h_n = \sum_{1 \leq i_1 \leq i_2 \leq \cdots \leq i_n} x_{i_1}x_{i_2}\cdots x_{i_n}.
\end{equation}
Given a partition \(\lambda \in \Par\), define \(h_\lambda = h_{\lambda_1}h_{\lambda_2}\cdots h_{\lambda_{\ell(\lambda)}}\).

\begin{theorem} \label{thm:hn_sym}
		The four following statements are true:
		\begin{enumerate}
				\item \label{item:hl_symd_basis}
							The set \(\{h_\lambda : \lambda \in \Par(d)\}\) forms a basis for \(\Sym_d\).
				\item \label{item:hl_sym_basis}
							The set \(\{h_\lambda : \lambda \in \Par\}\) forms a basis for \(\Sym\).
				\item \label{item:hn_sym_generates}
							The set \(\{h_i : i \in \positives\}\) generates \(\Sym\) and is algebraically independent.
				\item \label{item:hn_sym_graded_isom}
							The map \(N' \to \Sym\) obtained by the extension of the map \(x_i \mapsto h_i\) is an isomorphism of graded \(\rationals\)-algebras.
		\end{enumerate}
\end{theorem}

\begin{example}[Theorem~\ref{thm:hn_sym}\ref{item:hl_symd_basis} for \(d = 3\)]
    We claim that \(\{h_{\lambda} : \lambda \in \Par(3)\}\) is a basis for \(\Lambda_3\).
    \[
        \begin{bsmallmatrix}
            h_{1,1,1} \\
            h_{2,1}   \\
            h_3
        \end{bsmallmatrix}
        =
        \begin{bsmallmatrix}
            6 & 3 & 1 \\
            3 & 2 & 1 \\
            1 & 1 & 1
        \end{bsmallmatrix}
        \begin{bsmallmatrix}
            m_{1,1,1} \\
            m_{2,1}   \\
            m_3
        \end{bsmallmatrix}.
    \]
    Since the matrix
		\(
        \begin{bsmallmatrix}
            6 & 3 & 1 \\
            3 & 2 & 1 \\
            1 & 1 & 1
        \end{bsmallmatrix}
    \)
    is invertible, \(\{h_{1,1,1}, h_{2,1}, h_3\}\) forms a basis of \(\Lambda_3\). Moreover,
		\(
        \det\begin{bsmallmatrix}
            6 & 3 & 1 \\
            3 & 2 & 1 \\
            1 & 1 & 1
        \end{bsmallmatrix} = 1,
    \)
    so the matrix has integer entries and determinant \(1\), making it also a \(\mathbb{Z}\)-basis.
		However, the eigenvalues of
		\(
				\begin{bsmallmatrix}
						6 & 3 & 1 \\
						3 & 2 & 1 \\
						1 & 1 & 1
				\end{bsmallmatrix}
		\)
		are \(1\), \(4 + \sqrt{15}\), and \(4 - \sqrt{15}\),
		so that is something weird about this matrix.
\end{example}

Given weak compositions \(\alpha, \beta\),
let \(N_{\alpha\beta}\) be the number of contingency tables with entries in \(\nonnegatives\) such that \(\operatorname{row}(A) = \alpha\) and \(\operatorname{col}(A) = \beta\).

We list some facts about \(N_{\alpha\beta}\), whose proofs are analogous to Facts~\ref{fact:Mab-same-sum}--\ref{cor:Mab-Mlm} for \(M_{\alpha\beta}\), and are omitted.
These facts are used later without reference.

\begin{fact}
	If \(\alpha, \beta\) are weak compositions of different integers, then \(N_{\alpha\beta} = 0\).
\end{fact}

\begin{fact}
	For any weak compositions \(\alpha, \beta\),
	\(N_{\alpha\beta} = N_{\beta\alpha}\).
\end{fact}

\begin{fact}
	If \(\alpha\) is a rearrangement of \(\alpha'\) and \(\beta\) is a rearrangement of \(\beta'\),
	then \(N_{\alpha\beta} = N_{\alpha'\beta'}\).
\end{fact}

\begin{corollary}
	If \(\alpha\) is a rearrangement of a partition \(\lambda\) and \(\beta\) is a rearrangement of a partition \(\mu\),
	then \(N_{\alpha\beta} = N_{\lambda\mu}\).
\end{corollary}

\begin{lemma} \label{lem:generating-function-N-contingency-tables-compositions}
	Let \(x = (x_1, x_2, \ldots)\) and \(y = (y_1, y_2, \ldots)\) be sequences of variables.
	Then,
	\begin{equation}
		\prod_{i, j \geq 1} (1 - x_iy_j)^{-1}
		= \sum_{\substack{\text{contingency tables } A \\ \text{with entries in } \nonnegatives}} \operatorname{wt}(A)
		= \sum_{\substack{\text{weak compositions} \\ \alpha, \beta}} N_{\alpha\beta} x^\alpha y^\beta.
	\end{equation}
\end{lemma}

\begin{proof}
	Note that each monomial in the expansion of the power series
	\begin{equation}
		\prod_{i, j \geq 1} (1 - x_iy_j)^{-1}
		= \prod_{i, j \geq 1} \left( (x_iy_j)^0 + (x_iy_j)^1 + (x_iy_j)^2 + \cdots \right)
	\end{equation}
	corresponds to a choice of a nonnegative integer for the exponent of each \(x_iy_j\), which corresponds to a contingency table with entries in \(\nonnegatives\).
	Therefore,
	\begin{equation}
		\prod_{i, j \geq 1} (1 - x_iy_j)^{-1}
		= \sum_{\substack{\text{contingency tables } A \\ \text{with entries in } \nonnegatives}} \operatorname{wt}(A) 
		= \sum_{\substack{\text{weak compositions} \\ \alpha, \beta}} N_{\alpha\beta} x^\alpha y^\beta, 
	\end{equation}
	where the last equation is obtained by computing the coefficient of each monomial.
\end{proof}

\begin{proposition} \label{prop:generating-function-N-contingency-tables-partitions}
	Let \(x = (x_1, x_2, \ldots)\) and \(y = (y_1, y_2, \ldots)\) be sequences of variables.
	Then,
	\begin{equation}
		\prod_{i, j \geq 1} (1 - x_iy_j)^{-1} = \sum_{\substack{\text{partitions} \\ \lambda, \mu}} N_{\lambda\mu} m_\lambda(x) m_\mu(y).
	\end{equation}
\end{proposition}

\begin{proof}
	We have
	\begin{equation}
		\prod_{i, j \geq 1} (1 - x_iy_j)^{-1}
		= \sum_{\substack{\text{weak compositions} \\ \alpha, \beta}} N_{\alpha\beta} x^\alpha y^\beta
		= \sum_{\substack{\text{partitions} \\ \lambda, \mu}} N_{\lambda\mu} m_\lambda(x) m_\mu(y),
	\end{equation}
	where the first equality is by Lemma~\ref{lem:generating-function-N-contingency-tables-compositions},
	and the second equality is obtained by collecting all monomials whose exponents are rearrangements of the same partition.
\end{proof}

\begin{proposition} \label{prop:hl-expansion-in-m-basis}
	Let \(\lambda, \mu \in \Par(d)\).
	Then,
	\begin{equation}
		h_\lambda = \sum_{\mu \in \Par(d)} N_{\lambda\mu}m_\mu.
	\end{equation}
\end{proposition}

\begin{proof}
	Note that 
	\begin{equation}
		h_k(x)
		= \sum_{\substack{\nonnegatives\text{-sequences } \alpha \\ \text{with sum } k}}
			x^\alpha
	\end{equation}
	Hence,
	\begin{align}
		h_\lambda(x)
		= \prod_{i\geq 1} h_{\lambda_i}
		&= \prod_{i\geq 1}
			\sum_{\substack{\text{weak compositions } \alpha \\ \text{with entries in } \nonnegatives \\ \text{and sum } \lambda_i}}
				x^{\alpha^{(i)}} \\
		&= \sum_{\substack{\text{contingency tables } A \\ \text{with entries in } \nonnegatives \\ \text{and column sum } \lambda}}
			x^{\operatorname{row}(A)}.
	\end{align}
	By Lemma~\ref{lem:generating-function-N-contingency-tables-compositions},
	\(h_\lambda(x)\) is the coefficient of \(y^\lambda\) in \(\prod_{i, j \geq 1} (1 - x_iy_j)^{-1}\) as a power series in \(y\) with coefficients in \(\rationals[[x_1, x_2, \ldots]]\),
	that is,
	\begin{equation}
		h_\lambda(x) = \sum_{\text{weak compositions } \alpha} N_{\alpha\lambda} x^\alpha = \sum_{\text{partitions } \mu} N_{\mu\lambda} m_\mu(x),
	\end{equation}
	which completes the proof.
\end{proof}

The following corollary is a consequence of Propositions~\ref{prop:generating-function-N-contingency-tables-partitions} and~\ref{prop:hl-expansion-in-m-basis}.

\begin{corollary}
	Let \(x = (x_1, x_2, \ldots)\) and \(y = (y_1, y_2, \ldots)\) be sequences of variables.
	Then,
	\begin{equation}
		\prod_{i, j \geq 1} (1 - x_iy_j)^{-1} = \sum_{\lambda \in \Par} h_\lambda(x) m_\lambda(y).
	\end{equation}
\end{corollary}

Recall that \(\Lambda = \mathbb{Q}[e_1, e_2, \ldots]\) is secretly a polynomial ring. Given any \(\mathbb{Q}\)-algebra \(A\) and \(a_1, a_2, \ldots \in A\), there exists a unique homomorphism \(\varphi: \Lambda \to A\) with \(\varphi(e_k) = a_k\).

Let \(\omega\) be the unique homomorphism \(\omega: \Lambda \to \Lambda\) such that \(\omega(e_k) = h_k\). Note that
\[
    \omega(e_{\lambda}) = \omega(e_{\lambda_1} e_{\lambda_2} \cdots e_{\lambda_{\ell(\lambda)}}) = h_{\lambda_1} h_{\lambda_2} \cdots h_{\lambda_{\ell(\lambda)}} = h_\lambda.
\]

\begin{theorem}
    \(\omega\) is an involution, that is, \(\omega \circ \omega = \operatorname{id}\). In particular, \(\omega\) is an isomorphism.
\end{theorem}

\begin{proof}
    (To be written later).
\end{proof}

\begin{proof}[Proof of Theorem~\ref{thm:hn_sym}(\ref{item:hl_symd_basis})]
    Since \(\omega\) is an isomorphism of \(\Sym\), and \(\omega(e_\lambda) = h_\lambda\), and \(\{e_\lambda : \lambda \in \Par(d)\}\) forms a basis for \(\Sym_d\) by Theorem~\ref{thm:en_sym}(\ref{item:el_symd_basis}), it follows that \(\{h_\lambda : \lambda \in \Par(d)\}\) forms a basis for \(\Sym_d\).
\end{proof}
