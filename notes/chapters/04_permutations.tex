\chapter{Permutations}

Let \(n\) be a non-negative integer.
The \vocab{\(n\)\textsuperscript{th} symmetric group} \(\sym{n}\)
is the group of bijections from \(\{1, 2, \ldots, n\}\) to itself,
under composition.
A permutation \(w \in \sym{n}\) can be written in \vocab{two-line notation} as
\[
    \begin{pmatrix}
        1    & 2    & \cdots & n    \\
        w(1) & w(2) & \cdots & w(n)
    \end{pmatrix},
\]
or in \vocab{one-line notation} as
\[
    w(1)w(2)\cdots w(n).
\]
We may also think of a permutation \(w \in \sym{n}\) as a directed graph with vertex set \(\{1, 2, \ldots, n\}\) and edges \(\{(i, w(i)) \mid i = 1, \ldots, n\}\). The connected components of this graph are cycles.


Each permutation \(w\) has an associated partition \(\cyc(w) \vdash n\) whose parts are the lengths of the cycles of \(w\), called the \vocab{cycle type} of \(w\). It is known that \(\cyc(w) = \cyc(u)\) if and only if \(w\) and \(u\) are conjugate in \(\sym{n}\). Thus, the conjugacy class corresponding to a partition \(\lambda \vdash n\) is
\[
    \zeta_\lambda = \{ w \in \sym{n} \mid \cyc(w) = \lambda \}.
\]

The centralizer of \(w \in \sym{n}\) is
\[
    Z(w) = \{ u \in \sym{n} \mid uw = wu \}.
\]
Let \(\lambda\) be a partition of \(n\), and let \(j_i\) be the number of parts of \(\lambda\) equal to \(i\). For \(w \in \zeta_\lambda\), the size of the centralizer is
\[
    |Z(w)| = \prod_{i=1}^n (j_i)! \prod_{k=1}^{\ell(\lambda)} \lambda_k,
\]
which does not depend on the choice of \(w \in \zeta_\lambda\). We denote this expression by \(z_\lambda\). Moreover, the size of the conjugacy class is
\[
    |\zeta_\lambda| = \frac{n!}{z_\lambda}.
\]

A permutation \(w \in \sym{n}\) is called a \vocab{transposition} if \(\cyc(w) = 2\,1^{n-2}\), meaning \(w\) swaps two elements and fixes the rest. A \vocab{simple transposition} is a transposition of the form \(\eltr{i} = (i, i+1)\) for some \(i \in \{1, \ldots, n-1\}\).

The symmetric group \(\sym{n}\) is generated by the simple transpositions \(\eltr{1}, \eltr{2}, \ldots, \eltr{n-1}\), subject to the relations:
\begin{align*}
    \eltr{i}^2                 & = 1,                                           \\
    \eltr{i}\eltr{j}           & = \eltr{j}\eltr{i} \quad \text{if } |i-j| > 1, \\
    \eltr{i}\eltr{i+1}\eltr{i} & = \eltr{i+1}\eltr{i}\eltr{i+1}.
\end{align*}
Groups with presentations similar to this one are called \vocab{Coxeter groups}.

Given a permutation \(w \in \sym{n}\), an \vocab{inversion} is a pair \((i, j)\) such that \(i < j\) and \(w(i) > w(j)\). The number of inversions of \(w\) is denoted by \(\inv(w)\). The \vocab{Coxeter length} of \(w\) is the minimum number of simple transpositions needed to express \(w\) and equals \(\inv(w)\).

The \vocab{sign} of a permutation \(w\) is defined as \(\sign(w) = (-1)^{\inv(w)}\). This function \(\sign \colon \sym{n} \to \{\pm 1\}\) is a group homomorphism. For \(w \in \zeta_\lambda\), we have \(\sign(w) = (-1)^{n - \ell(\lambda)}\), where \(\ell(\lambda)\) is the number of parts in \(\lambda\).

The symmetric group \(\sym{n}\) has a unique element of minimum inversion number, namely the identity permutation \(\id\) with \(\inv(\id) = 0\), and a unique element of maximum inversion number, the permutation \(\omega_0 = n\, (n-1)\, \ldots\, 1\) with \(\inv(\omega_0) = \binom{n}{2}\).

For \(w \in \sym{n}\), the \vocab{permutation matrix} \(M(w) \in \GL{n}(\complexes)\) is defined by \(M(w)_{ij} = \delta_{i, w(j)}\). The map \(M \colon \sym{n} \to \GL{n}(\complexes)\) is a group homomorphism, and \(\sign\) equals the composition \(\det \circ M\).

A \vocab{(linear) representation} of a group \(G\) is a group homomorphism \(\rho \colon G \to \GL{n}(\complexes)\) for some \(n\). For example, \(M \colon \sym{n} \to \GL{n}(\complexes)\) is a representation. A \vocab{permutation representation} of \(G\) is a homomorphism \(\rho \colon G \to \sym{n}\) for some \(n\). Given any permutation representation \(\rho \colon G \to \sym{n}\), composing with \(M\) yields a linear representation \(\rho' \colon G \to \GL{n}(\complexes)\).

Given a set \(S\) and a group action \(\rho \colon G \to \sym(S)\), the \vocab{fixed set} of \(S\) under \(\rho\) is
\[
    S^G = \{ s \in S \mid \rho(g)(s) = s \text{ for all } g \in G \}.
\]
Similarly, if \(V\) is a vector space with a representation \(\rho \colon G \to \GL{V}\), the \vocab{invariant subspace} is
\[
    V^G = \{ v \in V \mid \rho(g)v = v \text{ for all } g \in G \}.
\]
One can view \(V^G\) as the intersection of the eigenspaces associated with the eigenvalue \(1\) of the matrices \(\rho(g)\) for all \(g \in G\).

As an example, let \(\mathcal{G}_n\) be the set of all graphs on the vertex set \(\{1, 2, \ldots, n\}\). The group \(\sym{n}\) acts on \(\mathcal{G}_n\) by permuting vertices: for \(w \in \sym{n}\) and \(G \in \mathcal{G}_n\), the graph \(w \cdot G\) is obtained by applying \(w\) to the vertices of \(G\). Under this action, the only fixed graphs are the empty graph and the complete graph.

An \vocab{action of a group \(G\) on a ring \(R\)} is a group homomorphism \(\rho \colon G \to \Aut(R)\), where \(\Aut(R)\) is the group of ring automorphisms of \(R\). The fixed set \(R^G\) is a subring of \(R\).

An \vocab{algebra over a field \(k\)} is a vector space \(A\) over \(k\) equipped with a bilinear multiplication \(\cdot \colon A \times A \to A\). If \(A\) is a \(k\)-algebra, it contains a copy of \(k\) as a subring. Conversely, if \(A\) is a ring containing a field \(k\) as a subring, then \(A\) is a \(k\)-algebra.

Examples of \(k\)-algebras include the ring of polynomials \(k[x]\), the ring of power series \(k[[x]]\), and the ring of matrices \(\Mat_n(k)\).

Given \(k\)-algebras \(A\) and \(B\), a \vocab{\(k\)-algebra homomorphism} is a map \(\varphi \colon A \to B\) that is both a ring homomorphism and \(k\)-linear. An \vocab{action of a group \(G\) on a \(k\)-algebra \(A\)} is a group homomorphism \(\rho \colon G \to \Aut(A)\), where \(\Aut(A)\) is the group of \(k\)-algebra automorphisms of \(A\). The fixed set \(A^G\) is a \(k\)-subalgebra of \(A\).

A \vocab{monomial} is a product of powers of variables. In \(\rationals[x_1, x_2, \ldots, x_n]\), monomials are indexed by \(\mathbb{N}^n\):
\[
    x^\alpha = x_1^{\alpha_1}x_2^{\alpha_2}\cdots x_n^{\alpha_n},
\]
for \(\alpha = (\alpha_1, \alpha_2, \ldots, \alpha_n) \in \mathbb{N}^n\). The \vocab{degree} of \(x^\alpha\) is \(\deg(x^\alpha) = \alpha_1 + \alpha_2 + \cdots + \alpha_n\). The set of monomials forms a basis for \(\rationals[x_1, x_2, \ldots, x_n]\) as a vector space over \(\rationals\).

We denote by \(\rationals[x_1, x_2, \ldots, x_n]_d\) the subspace spanned by monomials of degree \(d\), giving the direct sum decomposition
\[
    \rationals[x_1, x_2, \ldots, x_n] = \bigoplus_{d=0}^\infty \rationals[x_1, x_2, \ldots, x_n]_d.
\]

The group \(\sym{n}\) acts on monomials by
\[
    w \cdot x^\alpha = x^{w(\alpha)} = x_{w^{-1}(1)}^{\alpha_1}x_{w^{-1}(2)}^{\alpha_2}\cdots x_{w^{-1}(n)}^{\alpha_n}.
\]
This action extends linearly to \(\rationals[x_1, x_2, \ldots, x_n]\).
The \vocab{algebra of symmetric polynomials} is the subalgebra \(\Sym(n) = \rationals[x_1, x_2, \ldots, x_n]^{\sym{n}}\) consisting of all polynomials fixed under this action.
Let \(\Sym(n)_d = \Sym(n) \cap \rationals[x_1, x_2, \ldots, x_n]_d\).

A \(k\)-algebra \(A\) is \vocab{graded} if
there are \(k\)-linear subspaces \(A_d\), for \(d \in \nonnegatives\),
such that
\begin{equation*}
    A = \bigoplus_{d \in \nonnegatives} A_d
\end{equation*}
and the restriction of the multiplication map \(A \times A \to A\) to \(A_d \times A_e\) has image in \(A_{d+e}\) for all \(d, e \geq 0\).

For example, the \(\rationals\)-algebra of polynomials \(\rationals[x_1, x_2, \ldots, x_n]\) is graded, where the homogeneous component of degree \(d\) is \(\rationals[x_1, x_2, \ldots, x_n]_d\).
Similarly, the \(\rationals\)-algebra of symmetric polynomials \(\Sym(n)\) is graded, where the homogeneous component of degree \(d\) is \(\Sym(n)_d\).

If \(A\) is a graded \(k\)-algebra and each \(A_d\) is finite-dimensional,
then the \vocab{Hilbert series} of \(A\) is the formal power series
\begin{equation*}
    \Hilb(A; t) = \sum_{d=0}^\infty \dim_k(A_d)t^d.
\end{equation*}

For \(\rationals[x_1, x_2, \ldots, x_n]\),
the dimension of the homogeneous component \(\rationals[x_1, x_2, \ldots, x_n]_d\) is
\begin{equation*}
    \dim_k(\rationals[x_1, x_2, \ldots, x_n]_d) = \binom{n+d-1}{d}.
\end{equation*}
Thus, the Hilbert series of \(\rationals[x_1, x_2, \ldots, x_n]\) is
\begin{equation*}
    \Hilb(\rationals[x_1, x_2, \ldots, x_n]; t) = \sum_{d=0}^\infty \binom{n+d-1}{d}t^d = \frac{1}{(1-t)^n} = (1-t)^{-n}.
\end{equation*}

Given a weak composition \(\alpha = \composition{\alpha_1, \alpha_2, \ldots, \alpha_n}\),
let \(\sort{a}\) be the partition obtained by sorting the parts of \(\alpha\) in decreasing order.

For example, \(\sort{\composition{2, 0, 3, 1, 3}} = \composition{3, 3, 2, 1}\).
Note that, if \(f \in \Sym(n)\),
then if \(\alpha, \beta\) are compositions such that \(\sort{\alpha} = \sort{\beta} = \lambda\),
we have \([x^\alpha]f = [x^\beta]f = [x^\lambda]f\).

For each partition \(\lambda \in \Par_{n, \infty}\),
we define the \vocab{monomial symmetric polynomial \(m_\lambda \in \Sym(n)\)} by
\begin{equation*}
    m_\lambda = \sum_{\sort{\alpha} = \lambda} x^\alpha.
\end{equation*}

For example, for \(n = 3\),
\begin{align*}
    m_{\composition{4, 2, 0}} & = x_1^4x_2^2 + x_1^2x_2^4 + x_1^4x_3^2 + x_1^2x_3^4 + x_2^4x_3^2 + x_2^2x_3^4, \text{ and} \\
    m_{\composition{1, 1, 1}} & = x_1x_2x_3.
\end{align*}

The set of monomial symmetric polynomials
\[
    \{ m_\lambda \mid \lambda \in \Par_{n, \infty}(d) \}
\]
is a basis for \(\Sym(n)_d\).
As a corollary, the set of monomial symmetric polynomials
\[
    \{ m_\lambda \mid \lambda \in \Par_{n, \infty} \}
\]
is a basis for \(\Sym(n)\).
As another corollary, the Hilbert series of \(\Sym(n)\) is
\[
    \Hilb(\Sym(n); t) = \prod_{i=1}^n \frac{1}{1-t^i}.
\]

As a remark,
if we let \(B\) be the \(\rationals\)-algebra \(\rationals[x_1, x_2, \ldots, x_n]\)
with the nonstandard grading where \(\deg(x_i) = i\) for all \(i \in \interval{n}\),
then the Hilbert series of \(B\) is
\[
    \Hilb(B; t) = \prod_{i=1}^n \frac{1}{1-t^i}.
\]
Therefore, the homogeneous components of \(\Sym(n)\) and \(B\) are isomorphic as vector spaces.
Indeed, the set
\[
    \left\{
    x_\lambda = prod_{i=1}^n x_{\lambda_i} \mid \lambda \in \Par_{n, \infty}(d)
    \right\}
\]
is a basis for \(B_d\), also indexed by partitions of \(d\) with at most \(n\) parts.
Sadly (or not), the map \(x_\lambda \mapsto m_\lambda\) is not an isomorphism of algebras,
as multiplication is not preserved.
Can we find a better basis for \(\Sym(n)\)?