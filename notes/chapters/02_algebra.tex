\chapter{Formal Power Series}

In this chapter, we introduce formal power series.

To remove possible ambiguities,
note that
rings are assumed to have a multiplicative identity,
rings are not necessarily commutative, and
the multiplicative identity of a subring is the same as the multiplicative identity of the parent ring.

Let \(R\) be a ring.
As a set, let \(R[[x]]\) be \(R^{\nonnegatives}\), the set of all infinite sequences of elements of \(R\) indexed by the nonnegative integers.
In other words, \(R[[x]]\) is the set of all sequences \((a_n)_{n=0}^\infty\) where each \(a_n \in R\).

Given two sequences \((a_n)_{n=0}^\infty\) and \((b_n)_{n=0}^\infty\) in \(R[[x]]\), we define their sum as
\begin{equation}
    (a_n)_{n=0}^\infty + (b_n)_{n=0}^\infty = (a_n + b_n)_{n=0}^\infty
\end{equation}
and their product as
\begin{equation}
    (a_n)_{n=0}^\infty \cdot (b_n)_{n=0}^\infty = \left( \sum_{j=0}^n a_j b_{n-j} \right)_{n=0}^\infty.
\end{equation}
Such product is known as the \vocab{Cauchy product} of the two sequences.
With these operations, \(R[[x]]\) is a ring with additive identity \((0, 0, 0, \dots)\) and multiplicative identity \((1, 0, 0, \dots)\).

The ring \(R\) is embedded in \(R[[x]]\) via the map \(r \mapsto (r, 0, 0, \dots)\).
This allows us to view \(R\) as a subring of \(R[[x]]\).
Moreover, the polynomial ring \(R[x]\) is embedded in \(R[[x]]\) via the map \(r_0 + r_1 x + \cdots + r_n x^n \mapsto (r_0, r_1, \dots, r_n, 0, 0, \dots)\).
This allows us to view \(R[x]\) as a subring of \(R[[x]]\).

The embedding of \(R[x]\) into \(R[[x]]\) motivates the use of a similar notation for elements of \(R[[x]]\).
We designate the sequence \(A = (a_n)_{n=0}^\infty\) by the formal expression
\begin{equation*}
    A(x) = \sum_{n=0}^\infty a_n x^n,
\end{equation*}
even though such expression is not formed by the operation of addition in \(R[[x]]\).
We write \([x^k]A(x)\) to denote \(a_k\).

This notational convention allows for convenient reformuations of the definitions of addition and multiplication in \(R[[x]]\), given by
\begin{equation*}
    \sum_{n=0}^\infty a_n x^n + \sum_{n=0}^\infty b_n x^n = \sum_{n=0}^\infty (a_n + b_n) x^n
\end{equation*}
and
\begin{equation*}
    \left( \sum_{n=0}^\infty a_n x^n \right) \cdot \left( \sum_{n=0}^\infty b_n x^n \right) = \sum_{n=0}^\infty \left( \sum_{j=0}^n a_j b_{n-j} \right) x^n,
\end{equation*}
which are convenient, but one must be careful with the distinction between the formal summation and actual addition in \(R[[x]]\).

Unless otherwise relevant, the formal power series ring \(R[[x]] = R^{\nonnegatives}\) is equipped with the product topology where each copy of \(R\) is equipped with the discrete topology.

Let \(A_1(x), A_2(x), \dots\) be a sequence of formal power series,
and let \(A(x)\) be another formal power series.
The topology on \(R[[x]]\) implies that
the sequence \(A_1(x), A_2(x), \dots\) \vocab{converges to} \(A(x)\)
if and only if,
for every \(k\),
there exists \(N\) such that
for all \(n \geq N\),
we have \([x^k] A(x) = [x^k] A_n(x)\).
If the sequence converges to \(A(x)\),
we write \(A(x) = \lim_{n \to \infty} A_n(x)\).

\begin{example}
    Let \(A_i(x) = \sum_{j \geq i} x^j\).
    Then, \(\lim_{i \to \infty} A_i(x) = 0\).
\end{example}

Let \(A_1(x), A_2(x), \dots\) be a sequence of formal power series.
The \vocab{infinite sum} of the sequence is
\begin{equation}
    \sum_{i=1}^\infty A_i(x) = \lim_{n \to \infty} \sum_{i=1}^n A_i(x),
\end{equation}
which may or may not converge.
The \vocab{infinite product} of the sequence is
\begin{equation}
    \prod_{i=1}^\infty A_i(x) = \lim_{n \to \infty} \prod_{i=1}^n A_i(x),
\end{equation}
which may or may not converge.

Let \(A(x) = \sum_{i=0}^\infty a_i x^i\) and \(B(x) = \sum_{i=0}^\infty b_i x^i\).
Then, the \vocab{composition} of \(A(x)\) and \(B(x)\) is the formal power series
\begin{equation}
    A(B(x)) = \sum_{i=0}^\infty a_i B(x)^i,
\end{equation}
which may or may not converge.

\begin{proposition} \label{prop:composition_constant_term_zero}
    Let \(A(x), B(x) \in R[[x]]\) such that \([x^0] B(x) = 0\).
    Then, the composition \(A(B(x))\) is well-defined.
\end{proposition}

\begin{proposition}
    Let \(A(x) \in R[[x]]\) such that \(A(0) = 0\).
    Then, \(1 - A(x)\) is invertible in \(R[[x]]\), with inverse
    \begin{equation}
        \sum_{i=0}^\infty A(x)^i.
    \end{equation}
\end{proposition}

\begin{proof}
    Let \(B(x) = \sum_{i=0}^\infty x^i\).
    First, the expression \(\sum_{i=0}^\infty A(x)^i = B(A(x))\) is well-defined by Proposition~\ref{prop:composition_constant_term_zero}.
    Then, it is straightforward to verify that
    \begin{equation}
        (1 - A(x)) \cdot \sum_{i=0}^\infty A(x)^i = 1. \qedhere
    \end{equation}
\end{proof}