\chapter{Geometry}

First, a disclaimer:
we will set many conventions in this chapter,
and write some theorems,
but there's no guarantee that the theorems match the conventions we chose.

\section{Schubert Cells and Schubert Varieties}

Consider the vector space \(\complexes^n\) of \emph{column vectors} of length \(n\) with complex entries.
The vectors
\begin{equation}
    \mathbf{e}_1 = \begin{pmatrix} 1 \\ 0 \\ \vdots \\ 0 \end{pmatrix}, \quad
    \mathbf{e}_2 = \begin{pmatrix} 0 \\ 1 \\ \vdots \\ 0 \end{pmatrix}, \quad
    \ldots, \quad
    \mathbf{e}_n = \begin{pmatrix} 0 \\ 0 \\ \vdots \\ 1 \end{pmatrix}.
\end{equation}
form a basis for \(\complexes^n\).

Given column vectors \(\mathbf{c}_1, \ldots, \mathbf{c}_n \in \complexes^n\),
and row vectors \(\mathbf{r}_1, \ldots, \mathbf{r}_n \in \complexes^n\),
\begin{equation}
    \begin{pmatrix}
        \vert & \vert & & \vert \\
        \mathbf{c}_1 & \mathbf{c}_2 & \cdots & \mathbf{c}_n \\
        \vert & \vert & & \vert
    \end{pmatrix}
    \mathbf{e}_k = \mathbf{c}_k,
\end{equation}
\begin{equation}
    \mathbf{e}_k^T
    \begin{pmatrix}
        - & \mathbf{r}_1 & - \\
        - & \mathbf{r}_2 & - \\
        & \vdots & \\
        - & \mathbf{r}_n & -
    \end{pmatrix}
    = \mathbf{r}_k,
\end{equation}
and
\begin{equation}
    \begin{pmatrix}
        - & \mathbf{r}_1 & - \\
        - & \mathbf{r}_2 & - \\
        & \vdots & \\
        - & \mathbf{r}_n & -
    \end{pmatrix}
    \begin{pmatrix}
        \vert & \vert & & \vert \\
        \mathbf{c}_1 & \mathbf{c}_2 & \cdots & \mathbf{c}_n \\
        \vert & \vert & & \vert
    \end{pmatrix}
    =
    \begin{pmatrix}
        \mathbf{r}_1 \mathbf{c}_1 & \mathbf{r}_1 \mathbf{c}_2 & \cdots & \mathbf{r}_1 \mathbf{c}_n \\
        \mathbf{r}_2 \mathbf{c}_1 & \mathbf{r}_2 \mathbf{c}_2 & \cdots & \mathbf{r}_2 \mathbf{c}_n \\
        \vdots & \vdots & \ddots & \vdots \\
        \mathbf{r}_n \mathbf{c}_1 & \mathbf{r}_n \mathbf{c}_2 & \cdots & \mathbf{r}_n \mathbf{c}_n
    \end{pmatrix}
\end{equation}

Recall that the Grassmannian \(\Grassmannian_k(\complexes^n)\) is the set of \(k\)-dimensional subspaces of \(\complexes^n\).
We can obtain the Grassmannian by starting with the set of full-rank \(k \times n\) matrices, denoted by \(\mathcal{M}_{k \times n}^{\mathsf{FR}}\),
and quotient out by row operations, which, as we have seen, correspond to left multiplication by invertible matrices.
Hence,
\begin{equation}
    \Grassmannian_k(\complexes^n)
    \cong
    \GL{\complexes^k} \backslash \mathcal{M}_{k \times n}^{\mathsf{FR}}.
\end{equation}

Also, \(\GL{\complexes^n}\) act on the left on \(\complexes^n\),
so transitively \(\GL{\complexes^n}\) acts on the Grassmannian \(\Grassmannian_k(\complexes^n)\).

Fix \(X \in \Grassmannian_k(\complexes^n)\).
The Orbit--Stabilizer Theorem provides a bijection
\begin{equation}
    \left(\GL{\complexes^n}\right)_X \backslash \GL{\complexes^n} \to \Grassmannian_k(\complexes^n).
\end{equation}
Note that \(\GL{\complexes^n} / \left(\GL{\complexes^n}\right)_X \) is \emph{not} a group, because \( \left( \GL{\complexes^n} \right)_X \) is not a normal subgroup of \(\GL{\complexes^n}\).
Nevertheless, the quotient can still be seen as a set of left(?) cosets.

\begin{proposition}
    The stabilizer of \(\left\langle \mathbf{e}_1, \ldots, \mathbf{e}_k \right\rangle\) is the subgroup \(P_k\) of \(n \times n\) matrices of the form
    \begin{equation}
        \begin{pmatrix}
            A & B \\
            0 & C
        \end{pmatrix},
    \end{equation}
    where \(A\) is a \(k \times k\) invertible matrix, \(B\) is a \(k \times (n-k)\) matrix, and \(C\) is an \((n-k) \times (n-k)\) invertible matrix.
\end{proposition}

The set \(P_k\) is a \(k\)\textsuperscript{th} maximal parabolic subgroup of \(\GL{\complexes^n}\).
Note that ``parabolic'' means block upper triangle, and ``maximal'' means that there are only two blocks.
Remember, \(P_k\) is a subgroup of \(\GL{\complexes^n}\), but it's a normal subgroup of \(\GL{\complexes^n}\).

Hence, now we have
\begin{equation}
    \Grassmannian_k(\complexes^n) \cong P_k \backslash \GL{\complexes^n}.
\end{equation}

Therefore, the ``manifold'' dimension of \(\Grassmannian_k(\complexes^n)\) is
the difference between the ``manifold'' dimension of \(\GL{\complexes^n}\) and the ``manifold'' dimension of \(P_k\),
namely
\begin{equation}
    \dim(\Grassmannian_k(\complexes^n)) = n^2 - (k^2 + (n-k)k + (n-k)^2) = k(n-k).
\end{equation}

Let \( G = \GL{\mathbb{C}^n} \). Given a choice of basis \(\mathbf{e}_1, \ldots, \mathbf{e}_n\), we can describe specific subgroups of \( G \):
\begin{itemize}
    \item The Borel subgroup \( B = B^+ \) is the subgroup of invertible upper triangular matrices.
    \item The opposite Borel subgroup \( B^- \) is the subgroup of invertible lower triangular matrices.
    \item The intersection \( T = B^+ \cap B^- \) is the subgroup of invertible diagonal matrices, which is a maximal algebraic torus in \( G \). A maximal algebraic torus is a maximal abelian subgroup.
\end{itemize}

Without fixing a choice of basis,
a Borel subgroup of \( G \) is a maximal connected solvable subgroup.
Given a Borel subgroup \( B = B^+ \) of \( G \),
there exists an opposite Borel subgroup \( B^- \) is the unique Borel subgroup such that \( B \cap B^- = T \) is a maximal algebraic torus.

Let \(V = \complexes^n\).
Let \(P = P_k\).
Recall that \(P\) left-acts on \(G\), which gives a set of left cosets \(P \backslash G \cong \Grassmannian_k(V)\).
In turn, \(G\) right-acts on \(P \backslash G \cong \Grassmannian_k(V)\),
and consequently the subgroups \(B\) and \(T\) act on \(P \backslash G\).

\begin{proposition}
    The set of fixed points in \(P \backslash G \cong \Grassmannian_k(V)\) under the action of \(T\) is (congruent to) the set of \(binom{n}{k}\) 
    \(k\)-dimensional subspaces of \(V\) whose reduced row echelon form representative only has non-zero entries in the pivot positions.
    In other words,
    the set of fixed points are the vector spaces spanned by \(k\) basis vectors.
    In other words,
    the set of fixed points are the \(\binom{n}{k}\) \(k\)-dimensional subspaces of \(V\) of the form 
    \begin{equation}
        \left\langle \mathbf{e}_{i_1}, \ldots, \mathbf{e}_{i_k} \right\rangle,        
    \end{equation}
    where \(1 \leq i_1 < i_2 < \ldots < i_k \leq n\).
\end{proposition}

More explicitly, a \(T\)-fixed point in \(\Grassmannian_k(V)\) is a \(k\)-dimensional subspace that, in row echelon form, looks like
\begin{equation}
    \begin{bmatrix}
        1 & 0 & 0 & 0 & 0 & 0 & 0 \\
        0 & \textcolor{red}{0} & 1 & 0 & 0 & 0 & 0 \\
        0 & \textcolor{red}{0} & 0 & 1 & 0 & 0 & 0 \\
        0 & \textcolor{red}{0} & 0 & 0 & \textcolor{red}{0} & 1 & 0
    \end{bmatrix},
\end{equation}
and, in this example, we call it \(\mathbf{e}_{\composition{2, 1, 1, 0}}\).

\begin{question}
    What are the \(B\)-orbits? What are the \(B\)-orbits of the \(T\)-fixed points?
\end{question}

Given some \(T\)-fixed point, which is an arrangement of pivot positions, the \(B\)-orbit of that \(T\)-fixed point is all matrices that have the same pivot positions --- but now not necessarily zero entries everywhere else.
For example, the \(B\)-orbit of the above \(T\)-fixed point is
\begin{equation}
    \begin{bmatrix}
        1 & \ast & 0 & 0 & \ast & 0 & \ast \\
        0 & 0    & 1 & 0 & \ast & 0 & \ast \\
        0 & 0    & 0 & 1 & \ast & 0 & \ast \\
        0 & 0    & 0 & 0 & 0    & 1 & \ast
    \end{bmatrix}
\end{equation}
and the set of all such matrices is the Schubert cell \(X^\circ_{\composition{2, 1, 1, 0}}\).

In general, \(X^{\circ}_{\lambda}\) has pivot positions \(\lambda_k + 1, \lambda_{k-1} + 2, \ldots, \lambda_1 + k\).
We can rewrite this as
\begin{equation}
    X^{\circ}_{\lambda} =
    \left\{
        \begin{array}{c}
            V \in \Grassmannian_k(V) : \\
            \dim(V \cap \left\langle \mathbf{e}_1, \ldots, \mathbf{e}_{r} \right\rangle)
            =
            \left|[r] \cap \{ \lambda_k + 1, \lambda_{k-1} + 2, \ldots, \lambda_1 + k \} \right|
        \end{array}
    \right\}.
\end{equation}

The \vocab{Schubert variety} \(X_{\lambda}\) is
\begin{equation}
    X_\lambda =
    \left\{
        \begin{array}{c}
            V \in \Grassmannian_k(V) : \\
            \dim(V \cap \left\langle \mathbf{e}_1, \ldots, \mathbf{e}_{r} \right\rangle)
            \leq
            \left|[r] \cap \{ \lambda_k + 1, \lambda_{k-1} + 2, \ldots, \lambda_1 + k \} \right|
        \end{array}
    \right\}.
\end{equation}

Note that this condition gets less stricter as \(\lambda_k, \ldots, \lambda_1\) increase.
More precisely, we have
\begin{equation}
    X_\lambda = \bigcup_{\substack{\mu \supseteq \lambda \\ \mu \text{ inside } k \times (n-k)}} X^{\circ}_{\mu}.
\end{equation}

Note that this definition only depends on the sequence of vector spaces
\begin{equation}
    \varnothing \subset \left\langle e_1 \right\rangle \subset \left\langle e_1, e_2 \right\rangle \subset \ldots \subset \left\langle e_1, \ldots, e_n \right\rangle.
\end{equation}

\begin{definition}
    A \vocab{complete flag} in \(V\) is a sequence of vector spaces
    \begin{equation}
        \varnothing = F_0 \subset F_1 \subset \ldots \subset F_n = V
    \end{equation}
    such that \(\dim(F_i) = i\).
    \end{definition}

The \vocab{Schubert cell} and \vocab{Schubert variety} with respect to a complete flag \(F_{\bullet}\) are 
\begin{equation}
    X^{\circ}_{\lambda}(F_{\bullet}) =
    \left\{ V \in \Grassmannian_k(V) : \dim(V \cap V_r) = \left|[r] \cap \{ \lambda_k + 1, \lambda_{k-1} + 2, \ldots, \lambda_1 + k \} \right| \right\}
\end{equation}
and
\begin{equation}
    X_{\lambda}(F_{\bullet}) =
    \left\{ V \in \Grassmannian_k(V) : \dim(V \cap V_r) \leq \left|[r] \cap \{ \lambda_k + 1, \lambda_{k-1} + 2, \ldots, \lambda_1 + k \} \right| \right\},
\end{equation}

\section{Schubert Cohomology}

Details are sparse.
Be careful.
We will think of \(X\) as the Grassmannian \(\Grassmannian_k(\complexes^n)\),
but some of the following may be more general --- although we will not explicitly say so.

If \(X\) is a \emph{nice} topological space,
then the cohomology ring \(H^*(X)\) is an algebra over \(\mathbb{Z}\).
If \(Y \subset X\),
we get a corresponding class \([Y] \in H^*(X)\).
If \(G\) acts \(X\) transitively and \(g \in G\),
then \([g \cdot Y] = g \cdot [Y]\).
If \(Y, Z \subset X\) \emph{intersect transversally},%
\footnote{Two submanifolds of a given finite-dimensional smooth manifold are said to intersect transversally if at every point of intersection, their separate tangent spaces at that point together generate the tangent space of the ambient manifold at that point.}
then
\begin{align}
    [Y \cap Z] &= [Y] \cdot [Z], \\
    [Y \cup Z] &= [Y] + [Z].
\end{align}
However, if they don't intersect transversally, then you can move \(Y\) (using the group action) and move it to something in the same class so that it does intersect transversally, so there's some way to get around this.

If \(\varnothing = X_0 \subset X_1 \subset \cdots \subset X_d = X\) has each \(X_i\) being closed and each \(X_i \setminus X_{i-1}\) is a finite disjoint union on complex Euclidean spaces,
then the closures of these Euclidean spaces give a basis for the comohomology.

\begin{proposition}
    If \(\varnothing = X_0 \subset X_1 \subset \cdots \subset X_d = X\)
    has each \(X_i\) being closed and each \(X_i \setminus X_{i-1}\) is a finite disjoint union of complex affine spaces \(\mathbb{C}^{i-1}\),
    then the closures of these Euclidean spaces give an additive basis for the cohomology algebra \(H^*(X)\).
\end{proposition}

For example, consider \(X_0 = \varnothing\), \(X_1\) the singleton of the north pole \(p\), and \(X_2\) the Riemann sphere \(S^2\).
Then, \(H^*(X)\) is generated by the class of the north pole and the class of the south pole.
Therefore, every class in \(H^*(X)\) can be written as \(a [p] + b [S^2]\).
We can compute the product by computing the intersection of our basis elements in generic position (i.e.\ transversal intersection).
We get \([p] \cdot [p] = 0\), \([p] \cdot [S^2] = [S^2] \cdot [p] = 1\), and \([S^2] \cdot [S^2] = [S^2]\).

For example, \(\Grassmannian_k(\complexes^n)\) is a finite union of Schubert cells \(X^{\circ}_{\lambda}\).
Since the closure of a Schubert cell \(X^{\circ}_{\lambda}\) is a Schubert variety \(X_{\lambda}\), by letting \(\sigma_{\lambda} = [X_{\lambda}]\),
\begin{equation}
    H^*(\Grassmannian_k(\complexes^n)) = \bigoplus_{\lambda} \mathbb{Z} \cdot \sigma_{\lambda}.
\end{equation}

\begin{theorem}
    Let \(\lambda, \mu\) be partitions inside \(k \times (n-k)\).
    Then, in \(H^*(\Grassmannian_k(\complexes^n))\),
    \begin{equation}
        \sigma_{\lambda} \cdot \sigma_{\mu} = \sum_{\nu} c_{\lambda, \mu}^{\nu} \sigma_{\nu},
    \end{equation}
    where \(\nu\) ranges over partitions inside \(k \times (n-k)\) and \(c_{\lambda, \mu}^{\nu}\) are the Littlewood--Richardson coefficients.
\end{theorem}

If \(\lambda = \composition{p, 0, \cdots, 0}\), then \(X_\lambda\) is a \vocab{Pieri Schubert variety} or \vocab{special Schubert variety}, consisting of points in \(\Grassmannian_k(\complexes^n)\) of the form
\begin{equation}
    \left[\begin{array}{*{20}{c}}
        1 & 0 & \cdots & 0 & \ast & \ast & \cdots & \ast & 0 & \ast & \cdots & \ast \\
        0 & 1 & \cdots & 0 & \ast & \ast & \cdots & \ast & 0 & \ast & \cdots & \ast \\
        0 & 0 & \cdots & 1 & \ast & \ast & \cdots & \ast & 0 & \ast & \cdots & \ast \\
        0 & 0 & \cdots & 0 & 0    & 0    & \cdots & 0    & 1 & 0    & \cdots & 0
    \end{array}\right].
\end{equation}

For example, consider the Grassmannian \(\Grassmannian_1(\complexes^5)\),
\(\lambda = \composition{2}\) and \(\mu = \composition{1}\).
Then,
\begin{align}
    X^{\circ}_\lambda &=
    \left\{
    \begin{bmatrix}
        0 & 0 & 1 & \ast & \ast
    \end{bmatrix}
    \right\} \\
    X_\lambda &=
    \left\{
    \begin{bmatrix}
        0 & 0 & \ast & \ast & \ast
    \end{bmatrix}
    \right\} \\
    X^{\circ}_\mu(\text{reverse flag}) &=
    \left\{
    \begin{bmatrix}
        \ast & \ast & \ast & 1 & 0 
    \end{bmatrix}
    \right\} \\
    X_\mu(\text{reverse flag}) &=
    \left\{
    \begin{bmatrix}
        \ast & \ast & \ast & \ast & 0
    \end{bmatrix}
    \right\}.
\end{align}
We chose to write the reverse flag because it's the spaces intersect transversally.
From this, we can compute
\begin{equation}
    \sigma_{\lambda} \cdot \sigma_{\mu} = [X_\lambda \cap X_\mu(\text{reverse flag})] =
    \left[
    \left\{
    \begin{bmatrix}
        0 & 0 & \ast & \ast & 0
    \end{bmatrix}
    \right\}
    \right] = \sigma_{\composition{3}}.
\end{equation}

You can attempt to do the same with \(\lambda = \mu = \composition{1}\) in \(\Grassmannian_2(\complexes^4)\) and find that
\begin{equation}
    \sigma_{\composition{1}} \cdot \sigma_{\composition{1}} = \sigma_{\composition{2}} + \sigma_{\composition{1, 1}},
\end{equation}
by making the intersection transversal.%
\footnote{Oliver tried to do this in class but didn't finish the computation.}

\section{Flag Varieties}

Recall that a \vocab{flag} in a finite-dimensional vector space \(V\) is a sequence of nested subspaces:
\begin{equation}
    \{0\} = F_0 \subset F_1 \subset F_2 \subset \cdots \subset F_n = V,
\end{equation}
where each \(F_k\) is a \(k\)-dimensional subspace of \(V\); that is, \(\dim(F_k) = k\) for each \(k = 1, 2, \dots, n\). This means that each \(F_k\) corresponds to a point in the Grassmannian \(\operatorname{Gr}_k(V)\), which parametrizes the \(k\)-dimensional subspaces of \(V\).

Recall that in the study of Grassmannians, we associated to each point \(F_k\) (i.e., each \(k\)-dimensional subspace) a \(k \times n\) matrix \(A\), whose rows form a basis for \(F_k\).

In the case of a complete flag, we can similarly associate to it an invertible \(n \times n\) matrix \(A\), where the first \(k\) rows span \(F_k\) for each \(k\).
However, this representation is not unique, because we can perform certain row operations without changing the subspaces \(F_k\).
Specifically, we can perform \emph{downward} row operations (operations that use lower-numbered rows to eliminate entries in higher-numbered rows) without changing the spans of the rows corresponding to the \(F_k\).

This leads to the identification:
\begin{equation}
    \operatorname{Flags}_n \cong B_- \backslash \operatorname{GL}_n(\mathbb{C}),
\end{equation}
where \(B_-\) is the subgroup of \(\operatorname{GL}_n(\mathbb{C})\) consisting of lower-triangular invertible matrices.
Essentially, the flag variety \(\operatorname{Flags}_n\) can be viewed as the quotient of \(\operatorname{GL}_n(\mathbb{C})\) by the left action of \(B_-\).

Moreover, since \(\operatorname{GL}_n(\mathbb{C})\) contains the Borel subgroup \(B\) (the subgroup of upper-triangular invertible matrices) and the maximal torus \(T\) (the subgroup of diagonal invertible matrices), and since these subgroups act on \(\operatorname{GL}_n(\mathbb{C})\) by right multiplication, they also act on \(\operatorname{Flags}_n\) on the right.

\begin{proposition}
    The \(T\)-fixed points in \(\operatorname{Flags}_n\) are precisely the flags represented by permutation matrices.
\end{proposition}

In other words, the action of the torus \(T\) on the flag variety \(\operatorname{Flags}_n\) has fixed points corresponding to the permutation matrices, which are the matrices with exactly one entry of \(1\) in each row and column, and zeros elsewhere.

We can further study the orbits of the Borel subgroup \(B\) on \(\operatorname{Flags}_n\).

\begin{proposition}
    The \(B\)-orbits of a permutation matrix in \(\operatorname{Flags}_n\) are the sets of flags obtained by replacing all zeros that are not to the left or below a \(1\) with an arbitrary entry (denoted by \(\ast\)).
\end{proposition}

For example, consider the permutation matrix:
\begin{equation}
    \begin{bmatrix}
        0 & 0 & 1 \\
        1 & 0 & 0 \\
        0 & 1 & 0
    \end{bmatrix}.
\end{equation}
The \(B\)-orbit of this permutation matrix consists of all flags represented by matrices of the form:
\begin{equation}
    \begin{bmatrix}
        0 & 0 & 1 \\
        1 & \ast & 0 \\
        0 & 1 & 0
    \end{bmatrix},
\end{equation}
where the \(\ast\) represents an arbitrary complex number (since we are allowing arbitrary entries in positions not constrained by the permutation matrix and the action of \(B\)).
Similarly, for the identity permutation matrix
\begin{equation}
    \begin{bmatrix}
        1 & 0 & 0 \\
        0 & 1 & 0 \\
        0 & 0 & 1
    \end{bmatrix},
\end{equation}
the \(B\)-orbit consists of all flags represented by matrices of the form:
\begin{equation}
    \begin{bmatrix}
        1 & \ast & \ast \\
        0 & 1 & \ast \\
        0 & 0 & 1
    \end{bmatrix}.
\end{equation}

We call these sets of flags \vocab{Schubert cells} and denote them by \(X_w^\circ\), where \(w \in S_n\) is a permutation.
The index \(w\) corresponds to the permutation associated with the permutation matrix.%
\footnote{Oliver takes the transpose of the permutation matrix to index the Schubert cells, so be careful with this convention.}

Note that the flag variety \(\operatorname{Flags}_n\) maps naturally to the Grassmannian \(\operatorname{Gr}_k(\mathbb{C}^n)\) by sending a flag to its \(k\)-dimensional subspace \(F_k\).
This mapping allows us to relate the study of flags to the study of subspaces of fixed dimension.

By applying similar techniques as we did for the Grassmannian, we find that the cohomology ring \(H^*(\operatorname{Flags}_n)\) has an additive basis \(\{[X_w]\}\) indexed by permutations \(w \in S_n\). Here, \([X_w]\) denotes the cohomology class associated with the Schubert variety corresponding to \(w\).

Therefore, there exist coefficients \(c_{u,v}^w\) such that the product of two Schubert classes can be expressed as:
\begin{equation}
    [X_u] \cdot [X_v] = \sum_{w \in S_n} c_{u,v}^w [X_w].
\end{equation}
We call these coefficients the \vocab{Schubert structure constants}. They capture how the cohomology classes of Schubert varieties multiply in the cohomology ring.

In the case of Grassmannians, we had the Schur polynomials, which facilitated the computation of the structure constants. For the flag varieties, we have an analogous tool.

\begin{theorem}[Borel]
    \begin{equation}
        H^*(\operatorname{Flags}_n) \cong \mathbb{Z}[x_1, x_2, \ldots, x_n] \bigg/ \Sym_{n}^+,
    \end{equation}
    where \(\Sym_{n}^+\) denotes the ideal of symmetric polynomials with zero constant term.
\end{theorem}

That is, the cohomology ring of the flag variety is isomorphic to the polynomial ring in \(n\) variables modulo the ideal generated by the elementary symmetric polynomials of positive degree. This presentation, however, is not very convenient, as working with cosets modulo symmetric polynomials can be unwieldy.

To address this, we seek good representatives for these cosets.
The \vocab{Schubert polynomials} \(\mathfrak{S}_w\) provide such representatives.
By being representatives, they satisfy
\begin{equation}
    [\mathfrak{S}_u] \cdot [\mathfrak{S}_v] = \sum_{w \in S_n} c_{u,v}^w [\mathfrak{S}_w].
\end{equation}

However, the Schubert polynomials possess additional properties that make them particularly useful:

\begin{itemize}
    \item If \(w\) is a \(k\)-Grassmannian permutation corresponding to a partition \(\lambda\), then \(\mathfrak{S}_w = s_\lambda\).
    \item If \(\iota: S_n \hookrightarrow S_{n+1}\) is the natural inclusion, then \(\mathfrak{S}_{\iota(w)} = \mathfrak{S}_w\).
    \item The Schubert polynomials have positive monomial expansions.
    \item They satisfy the multiplication formula:
    \begin{equation}
        \mathfrak{S}_u \cdot \mathfrak{S}_v = \sum_{w \in S_n} c_{u,v}^w \mathfrak{S}_w,
    \end{equation}
    where the \(c_{u,v}^w\) are the Schubert structure constants.
\end{itemize}

These Schubert polynomials exist and are uniquely determined by these properties.

Moreover, the Schubert polynomials can be defined recursively. For the longest permutation \(w_0 \in S_n\), we have:
\begin{equation}
    \mathfrak{S}_{w_0} = x_1^{n-1} x_2^{n-2} \cdots x_{n-1}^1 x_n^0.
\end{equation}
For other permutations, we use divided difference operators to define the Schubert polynomials. Let \(\delta_i\) be the operator defined by:
\begin{equation}
    \delta_i(f) = \frac{f - s_i(f)}{x_i - x_{i+1}},
\end{equation}
where \(s_i\) is the simple transposition that swaps \(x_i\) and \(x_{i+1}\). The operator \(\delta_i\) applied to a polynomial \(f\) yields another polynomial.

For a permutation \(w\) that has a descent at position \(i\) (i.e., \(w(i) > w(i+1)\)), we define:
\begin{equation}
    \mathfrak{S}_{ws_i} = \delta_i(\mathfrak{S}_w).
\end{equation}
Although it may not be immediately apparent, it is a theorem that this recursive definition yields polynomials that are independent of the order in which descents are considered.

While Schubert polynomials provide a powerful tool,
and there are beautiful combinatorial formulas for them,
determining the structure constants \(c_{u,v}^w\) combinatorially remains a challenging problem.
This is one of the most important open problems in algebraic combinatorics.

Nevertheless, there are several cases where we have explicit combinatorial formulas for these structure constants:

\begin{itemize}
    \item If \(\mathfrak{S}_u = s_1\), then there is a combinatorial formula for \(c_{u,v}^w\) (Monk, 1959).
    \item If \(\mathfrak{S}_u = s_\lambda\) for a hook partition \(\lambda\), then there is a combinatorial formula for \(c_{u,v}^w\) (Sottile, 1996).
    \item If \(\mathfrak{S}_u = s_\lambda(x_1, x_2, \ldots, x_a)\) and \(\mathfrak{S}_v = s_\mu(x_1, x_2, \ldots, x_b)\), then there is a combinatorial formula for \(c_{u,v}^w\) (Purbhoo--Sottile, 2009).
    \item If the Schubert varieties are transverse without any perturbation, then there is a combinatorial formula for \(c_{u,v}^w\) (Knutson--Purbhoo, 2011).
    \item If \(u\) and \(v\) have descents restricted to the same two positions, then there is a combinatorial formula for \(c_{u,v}^w\) (Buch--Kresch--Purbhoo--Tamvakis, 2016).
    \item If \(u\) and \(v\) have descents restricted to the same three positions, then there is a combinatorial formula for \(c_{u,v}^w\) (Knutson--Zinn-Justin, 2017+).
    \item If all descents of \(u\) occur weakly before all the descents of \(v\), then there is a combinatorial formula for \(c_{u,v}^w\) (Huang, 2023).
    \item If \(u\) and \(v\) are inverses of Grassmannian permutations, then there is a combinatorial formula for \(c_{u,v}^w\) (Pechenik--Weigandt, 2024).
\end{itemize}

These results represent progress in understanding the multiplication of Schubert classes in the cohomology ring of the flag variety, although a general formula for the structure constants remains elusive.

Finally, there are several directions one can pursue to further explore this topic.
\begin{itemize}
    \item Replace \(\operatorname{GL}_n(\mathbb{C})\) with other groups \(G\), such as the orthogonal groups \(O_{2n}(\mathbb{C})\) and \(O_{2n+1}(\mathbb{C})\), or the symplectic group \(Sp_{2n}(\mathbb{C})\), and study the corresponding flag varieties.
    \item Replace ordinary cohomology \(H^*\) with other generalized cohomology theories, such as \(T\)-equivariant cohomology, quantum cohomology, or \(K\)-theory, to gain deeper insights into the topology and geometry of flag varieties and their Schubert subvarieties.
\end{itemize}